\documentclass[11pt]{article}
\usepackage{graphicx}
\usepackage{parskip}

\usepackage{xeCJK} 
\setCJKmainfont{Noto Sans CJK TC}

\usepackage{geometry}
\geometry{
 a4paper,
% total={210mm,297mm},
 left=1in,
 right=1in,
 top=1in,
 bottom=1in
}

\usepackage{fancyhdr}
\chead{\raisebox{0.7mm}{2017 NCTU Competitive Programming I - Practice Contest}}
\pagestyle{fancy}

\usepackage{csquotes}

\begin{document}

\pagenumbering{gobble}
\begin{center}
    {\LARGE Problem D}\\
    {\Large Inversion}\\
    {Time limit: 3 seconds}\\
    {Memory limit: 256 megabytes}
\end{center}

\textbf{\large Problem Description}

Given a positive integer $n$ and a sequence $s_1,\dots,s_n$ where $s_i$ is an
integer for each $i\in\{1,\dots,n\}$.
The inversion number of the sequence $s_1,\dots,s_n$ is the number of pairs
$(i,j)$ such that $s_i>s_j$ and $i<j$. For example, the inversion number of
the sequence $1,2,5,3,4$ is $2$ since the only two pairs $(3,4)$ and $(3,5)$
satisfy the criteria. Write a program to compute the inversion number.

\textbf{\large Input Format}

The input contains at most 25 test cases. 
Each case consists of two lines. The first line contains only an integer $n$. 
The second line consists of $n$ 32-bit signed integers 
$s_1,\dots,s_n$ separated by blanks.
You should compute the inversion numober of $s_1,\dots,s_n$.

\textbf{\large Output Format}

For each case, output the inversion number in one line.

\textbf{\large Sample Input}

\begin{verbatim}
2
5
1 2 5 3 4
5
1 2 5 4 3
\end{verbatim}

\textbf{\large Sample Output}
\begin{verbatim}
2
3
\end{verbatim}

\end{document}
