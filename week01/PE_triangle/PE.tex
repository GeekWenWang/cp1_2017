\documentclass[11pt]{article}
\usepackage{graphicx}
\usepackage{parskip}

\usepackage{xeCJK} 
\setCJKmainfont{Noto Sans CJK TC}

\usepackage{geometry}
\geometry{
 a4paper,
% total={210mm,297mm},
 left=1in,
 right=1in,
 top=1in,
 bottom=1in
}

\usepackage{fancyhdr}
\chead{\raisebox{0.7mm}{2017 NCTU Competitive Programming I - Practice Contest}}
\pagestyle{fancy}

\usepackage{csquotes}

\begin{document}

\pagenumbering{gobble}
\begin{center}
    {\LARGE Problem E}\\
    {\Large Triangles}\\
    {Time limit: 1 second}\\
    {Memory limit: 256 megabytes}
\end{center}

\textbf{\large Problem Description}

A triangle is a shape which can be formed by connecting three non-colinear
points with straight lines.
Given a retangular grid formed by $n$ horizontal lines and $m$ vertical lines.
How many different triangles be formed using the points on the intersections
of the grid if all grid cells have the same area?

\textbf{\large Input Format}

In first line of input, there is an integer $T$ ($T\le 100$) indicating the
number of test cases. Each of the following $n$ lines contains a test case.
Each test case has two positive integers $n$ and $m$ saparated by a blank. Both
of them are no more than $100$, and there prodect $nm$ is no more than $1024$.

\textbf{\large Output Format}

For each test case, output an integer representing the number of such triangles.

\textbf{\large Sample Input}

\begin{verbatim}
2
2 2
3 3
\end{verbatim}

\textbf{\large Sample Output}
\begin{verbatim}
4
76
\end{verbatim}

\end{document}
