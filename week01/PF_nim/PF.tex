\documentclass[11pt]{article}
\usepackage{graphicx}
\usepackage{parskip}

\usepackage{xeCJK} 
\setCJKmainfont{Noto Sans CJK TC}

\usepackage{geometry}
\geometry{
 a4paper,
% total={210mm,297mm},
 left=1in,
 right=1in,
 top=1in,
 bottom=1in
}

\usepackage{fancyhdr}
\chead{\raisebox{0.7mm}{2017 NCTU Competitive Programming I - Practice Contest}}
\pagestyle{fancy}

\usepackage{csquotes}

\begin{document}

\pagenumbering{gobble}
\begin{center}
    {\LARGE Problem F}\\
    {\Large Stone}\\
    {Time limit: 1 second}\\
    {Memory limit: 256 megabytes}
\end{center}

\textbf{\large Problem Description}

Alice and Bob are playing a game with some unbreakable stones. Initially there
are some piles of stones, and they'll take turns splitting a pile. In each turn,
a player should choose a pile with more than one stone and split it into at
least two equivalent piles. Alice move first, and the player who can't make a
valid move lose. Who will win the game if both of them play optimally?

\textbf{\large Input Format}

The first line contains an integer $T$ ($T\le 20$), the number of test cases.
Each test case contains two lines.
The first line contains an integer $n$ ($n\le 10000$), which is the number of
piles.
The second line contains $n$ integers $a_1,a_2,\dots,a_n$ ($a_i\le 10^9$),
giving the number of stones in each pile.

\textbf{\large Output Format}

For each test case, print ``Alice'' if Alice will win or ``Bob'' otherwise in
one line.

\textbf{\large Sample Input}

\begin{verbatim}
3
2
3 15
3
3 5 15
4 
3 5 15 21
\end{verbatim}

\textbf{\large Sample Output}
\begin{verbatim}
Alice
Alice
Bob
\end{verbatim}

\end{document}
