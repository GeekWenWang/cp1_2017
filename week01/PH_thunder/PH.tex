\documentclass[11pt]{article}
\usepackage{graphicx}
\usepackage{parskip}

\usepackage{xeCJK} 
\setCJKmainfont{Noto Sans CJK TC}

\usepackage{geometry}
\geometry{
 a4paper,
% total={210mm,297mm},
 left=1in,
 right=1in,
 top=1in,
 bottom=1in
}

\usepackage{fancyhdr}
\chead{\raisebox{0.7mm}{2017 NCTU Competitive Programming I - Practice Contest}}
\pagestyle{fancy}

\usepackage{csquotes}

\begin{document}

\pagenumbering{gobble}
\begin{center}
    {\LARGE Problem H}\\
    {\Large NCTU Thunder}\\
    {Time limit: 2 seconds}\\
    {Memory limit: 256 megabytes}
\end{center}

\textbf{\large Problem Description}

交⼤有一段時間參加 ACM-ICPC 的隊伍名稱都跟雷電有關:因為「交⼤都是雷」。
因此不可避免的,上場⽐賽難免也會雷雷的,交⼤的教練感到⾮常苦惱。參加 
ACM-ICPC 的隊伍,⼀隊需要三個⼈,⽽根據不禮貌的想像,⼀個隊伍「雷」的程度,
是由隊員兩兩之間的交互作⽤引發的「雷⼒」⼤⼩決定。譬如說 \verb+NCTU_Thor+ 
這⼀隊由 \verb+aaaaajack+、\verb+leopan0922+、\verb+iclan+ 三個⼈組成,
\verb+aaaaajack+ 跟 \verb+leopan0922+ 之間產⽣的「雷⼒」為 $a$、
\verb+leopan0922+ 跟 \verb+iclan+ 之間產⽣的「雷⼒」為 $b$、\verb+iclan+ 跟
\verb+aaaaajack+ 之間產⽣的「雷⼒」為 $c$,則 \verb+NCTU_Thor+ 這⼀隊的
「內部雷⼒」為 $a + b + c$,將與實際「雷」的程度成正⽐。

教練請⽰過校⾨⼝⼟地公後,總算知道學⽣與學⽣之間兩兩交互作⽤所
產⽣的雷⼒數值。請幫交⼤的教練寫⼀個程式,找出最理想的組隊⽅式,
也就是將所有隊伍的「總內部雷⼒」降到最低。

\textbf{\large Input Format}

第⼀⾏有⼀整數 $T$ 代表有多少測試資料,$T$ 最多 $20$。每⼀筆測試資料的
第⼀⾏有⼀個數字 $N$,代表有多少隊員,$N$ 必然被 $3$ 整除,且不⼤於 $21$。
⽅便起⾒,我們將參賽學⽣由 $1$ 到 $N$ 編號。接下來 $N$ ⾏中的第 $i$ ⾏,有
$N$ 個數字,其中第 $j$ 個就是記錄學⽣ $i$ 與學⽣ $j$ 之間的雷⼒ $R_{i,j}$。
測試資料中,雷⼒的數值必是不⼤於 $100$ 的⾮負整數、$R_{i,i} = 0$ 
且 $R_{i,j} = R_{j,i}$。如果不是這樣,出題⽼師就太雷了,這⼀題會送分。

\textbf{\large Output Format}

對每⼀個測試資料,輸出⼀⾏,內含⼀個數字,代表將隊伍分好後,可達
到的最低「總內部雷⼒」。

\textbf{\large Sample Input}

\begin{verbatim}
3
3
0 1 1
1 0 2
1 2 0
6
0 1 1 4 1 3
1 0 2 1 4 2
1 2 0 3 2 4
4 1 3 0 1 1
1 4 2 1 0 2
3 2 4 1 2 0
6
0 1 1 4 1 3
1 0 2 1 4 2
1 2 0 3 2 4
4 1 3 0 4 4
1 4 2 4 0 5
3 2 4 4 5 0
\end{verbatim}

\textbf{\large Sample Output}
\begin{verbatim}
4
8
11
\end{verbatim}

\end{document}
