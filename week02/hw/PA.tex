\documentclass[11pt]{article}
\usepackage{graphicx}
\usepackage{parskip}

\usepackage{xeCJK} 
\setCJKmainfont{Noto Sans CJK TC}

\usepackage{geometry}
\geometry{
 a4paper,
% total={210mm,297mm},
 left=1in,
 right=1in,
 top=1in,
 bottom=1in
}

\usepackage{fancyhdr}
\chead{\raisebox{0.7mm}{2017 NCTU Competitive Programming I - Homework}}
\pagestyle{fancy}

\usepackage{csquotes}

\begin{document}

\pagenumbering{gobble}
\begin{center}
    {\LARGE Problem A}\\
    {\Large PMA Expressions}\\
    {Time limit: 1 second}\\
    {Memory limit: 256 megabytes}
\end{center}

\textbf{\large Problem Description}

A Parenthesis-Multiplication-Addition (PMA) expression 
consists of only parentheses, multiplications, additions and digits. That is,
a PMA expression only uses \verb+(+, \verb+)+, \verb+*+ (multiplication), 
\texttt{+} (addition), and numbers.
To make the problem easier, we define valid PMA expressions for this problem 
as follows.
\begin{itemize}
\item A number is a valid PMA expression. For example, $4$ is a valid PMA 
expression.
\item If $expr$ is a valid PMA expression, then $(expr)$ is a valid PMA 
expression. For instance, $(4)$ is a valid PMA expression, since $4$ is. 
\item If $expr_1$ and $expr_2$ are valid PMA expressions, then both 
$expr_1+expr_2$ and $expr_1*expr_2$ are also valid. For instances, $1+2$ and 
$1*2$ are valid, since $1$ and $2$ are valid.
\item A valid PMA expression must be derived from the above rules.
\end{itemize}

Your task is to evaluate valid PMA expressions. You should output the results 
modulo $10^9+7$. You can implement the program with 64-bit integers.

\textbf{\large Input Format}

The input is terminated by end-of-file, and there are at most 100 test cases.
Each test case is a line consisting of \verb+(+, \verb+)+, \verb+*+,
\texttt{+}, and \verb+0+, \verb+1+, $\dots$, \verb+9+. For convenience, you
may assume any number in the input is a single digit. I.e., only 
$0,1,2,3,4,5,6,7,8,9$ are in the input. However, the intermidiate results might
be much larger. 

Note: all test cases are valid PMA expressions of length at most 2048.

\textbf{\large Output Format}

For each test case, output the evaluation result modulo $10^9+7$.

\textbf{\large Sample Input}

\begin{verbatim}
1+2*3+4
5*6*7*8*9
\end{verbatim}

\textbf{\large Sample Output}
\begin{verbatim}
11
15120
\end{verbatim}

\end{document}
