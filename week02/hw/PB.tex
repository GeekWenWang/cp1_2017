\documentclass[11pt]{article}
\usepackage{graphicx}
\usepackage{parskip}

\usepackage{xeCJK} 
\setCJKmainfont{Noto Sans CJK TC}

\usepackage{geometry}
\geometry{
 a4paper,
% total={210mm,297mm},
 left=1in,
 right=1in,
 top=1in,
 bottom=1in
}

\usepackage{fancyhdr}
\chead{\raisebox{0.7mm}{2017 NCTU Competitive Programming I - Homework}}
\pagestyle{fancy}

\usepackage{csquotes}

\begin{document}

\pagenumbering{gobble}
\begin{center}
    {\LARGE Problem B}\\
    {\Large Factoring a Function}\\
    {Time limit: 1 second}\\
    {Memory limit: 256 megabytes}
\end{center}

\textbf{\large Problem Description}

Given a function $f:\{1,\dots,n\}\rightarrow\{1,\dots,n\}$ mapping integers 
$1,\dots,n$ into values among $1,\dots,n$. We say $f$ can be factored
into two functions $g:\{1,\dots,n\}\rightarrow\{1,\dots,m\}$ and 
$h:\{1,\dots,m\}\rightarrow\{1,\dots,n\}$ if the following conditions hold.
\begin{itemize}
\item $m\le n$.
\item $h(g(x))=f(x)$ for $x\in\{1,\dots,n\}$.
\item $g(h(x))=x$ for $x\in\{1,\dots,m\}$.
\end{itemize}
For example, $f(x)=2$ for $n=3$ can be factored into $g(x)=1$ and $h(x)=2$ by
choosing $m=1$. 

If there does not exist such integer $m$, function $g$, and function $h$, then
we say that $f$ cannot be factored.



\textbf{\large Input Format}

Each test case consists of two lines. The first line is an integer $n$ 
($1\le n\le5$) indicating the domain and range of $f$. 
The second line consists of $n$
integers $f(1),\dots,f(n)$ saparated by blanks where $f(i)\in\{1,\dots,n\}$ for
$i\in\{1,\dots,n\}$.

The input is terminated by $n=0$, and there are at most 3456 test cases.

\textbf{\large Output Format}

For each test case, output \verb+yes+ if we can factor $f$ into some $g$ and 
$h$. Otherwise, output \verb+no+.

\textbf{\large Sample Input}

\begin{verbatim}
3
1 2 3
2
2 1
3
2 2 2
0
\end{verbatim}

\textbf{\large Sample Output}
\begin{verbatim}
yes
no
yes
\end{verbatim}

\end{document}
